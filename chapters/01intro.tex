\section{Introduction}\label{sec:introduction}
\begin{itemize}
    \item \textbf{Deep Learning} building hierarchical graph of concepts with many layers
    \begin{itemize}
        \item representations are expressed in terms of other, simpler representations
        \item \textbf{MLP}: function that maps input to output; composition of many simpler functions
    \end{itemize}
    \item \textbf{Knowledge Base} Apporach: hard-code knowledge or rules in formal language
    \item \textbf{Machine Learning}: the ability to extract ("learn") patterns from raw data
    \item \textbf{Representation Learning}: using machine learning to derive a representation (extract features); ex: autoencoders
\end{itemize}

\subsection{Who Should Read This Book?}
\begin{itemize}
    \item computer scientists
    \item some math background
    \item some programming experience
\end{itemize}

\subsection{Historical Trends in Deep Learning} 
\begin{itemize}
    \item \textbf{Cybernetics} (1940s-50s): aimed to computationally model the brain, very theoretical, very little learning mechanism
    \begin{itemize}
        \item \textbf{MCP Neuron}: first model of a neuron, inspired by human brain; used propositional logic, no learning mechanism
        \item \textbf{Perceptron}: first learning algorithm, used for binary classification; limited to linearly separable data
        \item \textbf{ADALINE}: special case of SGD
    \end{itemize}
    \item \textbf{Connectionism} (1980s-90s): introduced backpropagation, focus on MLPs and CNNs for automatic feature extraction on basic learning tasks
    \begin{itemize}
        \item \textbf{Backprop}: discovered independently in the 70s/80s by multiple groups; popularized by Rumelhart, Hinton, and Williams, efficient and scalable learning mechansim
        \item \textbf{MLP}: multi-layer perceptron; used for supervised learning; feature differentiable and continuous nonlinearities, which worked with backprop; universal approximator
        \item \textbf{CNN}: convolutional neural networks; used for image processing, introduced by LeCun et al. in 1989; uses local connectivity and weight sharing
    \end{itemize}
    \item \textbf{Deep Learning} (2000s-present): focus on large datasets, deeper models, new architectures, and computational power
    \begin{itemize}
        \item \textbf{GPU Computing}: use of graphics processing units to accelerate deep learning training
        \item \textbf{Transfer Learning}: leveraging pre-trained models on new tasks with limited data
        \item \textbf{Generative Models}: models that can generate new data samples, e.g., GANs and VAEs
    \end{itemize}
    \item Models became more useful as data sizes increased; performance increased despite very little difference in architecture
    \item Models became more complex with infrastructure improvements 
    \begin{itemize}
        \item faster CPUs, general purpose GPUs
        \item software libraries like TensorFlow, PyTorch, and JAX
    \end{itemize}
\end{itemize}